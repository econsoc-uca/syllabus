\documentclass[12]{article}
%\input{/home/grant/Dropbox/LaTeX/preamble} %% Rather use self-contained preamble below

%% LAYOUT AND TITLES
\usepackage{setspace}
\onehalfspacing
\usepackage[margin=1.1in]{geometry}
\setlength{\parindent}{0pt}
\setlength{\parskip}{10pt}
\usepackage{titling}
\newcommand{\subtitle}[1]{%
	\posttitle{%
		\par\end{center}
	\begin{center}\large#1\end{center}
	\vskip0.5em}%
}
%% Change title format to be more compact
%\usepackage{titling}
%\setlength{\droptitle}{-2em}
%  \title{Syllabus}
%  \pretitle{\vspace{\droptitle}\centering\huge}
%  \posttitle{\par}
%  \author{Grant R. McDermott}
%  \preauthor{\centering\large\emph}
%  \postauthor{\par}
%  \predate{\centering\large\emph}
%  \postdate{\par}
%  \date{}

%% FONTS
\usepackage[normalem]{ulem} %% For strikeout font: \sout()
\usepackage{lmodern}
\usepackage{amssymb,amsmath}
\usepackage{fontspec}
\usepackage{fontawesome}
\setfontfamily{\FA}{[FontAwesome.otf]}
% See: https://tex.stackexchange.com/a/50593
%\setmainfont[ 
%BoldFont       = texgyrepagella-bold.otf ,
%ItalicFont     = texgyrepagella-italic.otf ,
%BoldItalicFont = texgyrepagella-bolditalic.otf 
%]{texgyrepagella-regular.otf}
\setmainfont[
BoldFont       = FiraSans-SemiBold.otf ,
ItalicFont     = FiraSans-Italic.otf ,
BoldItalicFont = FiraSans-SemiBoldItalic.otf 
]{FiraSans-Regular.otf} %% /usr/share/texlive/texmf-dist/fonts/opentype/public/fira
\setmonofont[Mapping=tex-text]{inconsolata}	

%% MISC
\usepackage[colorlinks = true,
linkcolor = black,
urlcolor  = blue,
citecolor = blue,
anchorcolor = black]{hyperref}
\usepackage{tabularx}
\usepackage{booktabs}
\usepackage{hanging}
\usepackage{apacite}
\renewcommand{\refname}{Readings}


\begin{document}

\title{Economía Social y Humana}
\subtitle{\textsc{2019 syllabus}\vspace{-2ex}}
\author{Rony Rodriguez-Ramirez\\ Dept. Economía Aplicada, Universidad Centroamericana}
%\date{}  % Toggle commenting to test
\date{\vspace{-5ex}}

\maketitle

\section*{Sumario}

\begin{tabular}{ll} 
	\textbf{Cuándo:} 	& Lunes: 6 a 7:20 pm \\
	\textbf{Dónde:} 	& M-01 \\
	\textbf{Grupo:}		& B018 \\
	\textbf{Web:} 		& \href{https://github.com/...}{https://github.com/...} \\
	\textbf{Quién:} 	& Rony Rodriguez-Ramirez \\
						& \, \faMortarBoard 	\, Research Assistant \\
						& \, \faEnvelopeO 		\, \href{mailto:rony.maximiliano@doc.uca.edu.ni}{rony.maximiliano@doc.uca.edu.ni} \\
						& \, \faHourglassHalf 	\, Lunes: 7:30 a 8:30 pm  \\
\end{tabular} 

\section*{Descripción del curso}

Este seminario está dirigido a estudiantes de pregrado de economía aplicada y presentará temas actuales en economía
social y humana. Si bien es probable que algunos materiales se superpongan con sus otros cursos de 
la carrera, este no es solo otro curso teórico – práctico. Más bien, mi objetivo es proveerles con las temáticas actuales 
de economía social y humana, herramientas y técnicas prácticas que beneficiarán su conocimiento y trabajos futuros como 
economistas. Si  bien los conjuntos de datos y el enfoque de los materiales se vincularán predominantemente con la 
economía del desarrollo, las herramientas y los métodos se aplican ampliamente.


\newpage

\section*{Cuestiones prácticas}

\subsection*{Reglas de la clase y contrato didáctico}

Durante las clases presenciales los/las alumnos(as) deberán cumplir con las siguientes normas:
\begin{enumerate}
\item Realizar sesiones de consulta con el docente, en el día y hora establecidos para ello.
\item Cumplir con la puntualidad y asistencia a los seminarios que se impartan dentro de la clase.
\item Participar en clase y en todas las actividades orientadas por el docente durante ese período.
\item Atender y seguir las instrucciones del docente para la realización de cualquier trabajo o actividad dentro y fuera de clase.
\item Entregar las tareas y asignaciones orientadas por el docente en las fechas establecidas por él mismo.
\item Previo a cada clase, los estudiantes deberán haber leído el material instruido por el docente, el cual se detalla en el syllabus y en las agendas de trabajo. 
\item Mantener un ambiente de inclusión y respeto dentro de la clase.
\end{enumerate}
Durante las clases presenciales, el docente deberá cumplir con las siguientes normas:
\begin{enumerate}
\item Cumplimiento del horario de clases.
\item Entrega de notas en el tiempo correspondiente. 
\item Mantener un ambiente de inclusión y respeto dentro de la clase.
\item Atender las consultas establecidas por los/las alumnos(as). El único canal oficial de comunicación entre los/las alumnos(as) y el profesor será a través del correo institucional o por el Entorno Virtual de Aprendizaje. 	
\end{enumerate}

\vspace{-0.25cm}
\subsection*{Requirimientos de software}
Para algunas de las actividades tendremos que ocupar los siguientes softwares: \textit{\textbf{Stata}} y \textit{\textbf{R}}. Ya que uno de los propósitos de la clase es disminuir los costos, les recomiendo utilizar el lenguage de progrmación estadístico \textbf{\textit{R}} (descargar \href{https://www.r-project.org/}{aquí}). Por favor, asegurense de instalar \textbf{RStudio IDE} también (descargar \href{https://www.rstudio.com/products/rstudio/download/preview/}{aquí}).

\vspace{-0.25cm}
\section*{Evaluación y calificación}

\subsection*{Determinación de notas}


Las notas serán determinadas de la siguiente manera: 

\begin{table}[!h] \centering 
	%\caption{\textsc{grades} }
	\label{tab:grades} 
	\begin{tabularx}{0.5\textwidth}{Xr} 
		\toprule
		%		\multicolumn{2}{c}{EC 607}  \\
		%		\midrule
		2 $ \times$ Ensayos individuales (10\% cada uno)	& 20\% \\
		1 $ \times$ Ensayo grupal (15 \% cada uno)			& 15\% \\
		2 $ \times$ Talleres prácticos					& 40\% \\
		1 $ \times$ Presentación corta					& 15\% \\
		Participación en clase							& 10\% \\
		\bottomrule
	\end{tabularx} 
\end{table} 

Este desglose debería (con suerte) ser bastante autoexplicativo. Cualquier requisito específico se aclarará a medida que avanzamos en el curso. Sin embargo, aquí hay algunos detalles adicionales para \sout{pedantes} las personas a quienes les gusta todo lo que está escrito con precisión:

\vspace{-0.25cm}
\subsubsection*{Presentación corta}

La mayoría de las conferencias tienen dos o más lecturas clave; consulte el esquema de la bibliografía final de este
documento. La presentación corta se hará en grupos de 3 a 4 personas, y será una versión resumida de los artículos
académicos o capítulos de libros (10 a 15 minutos máximo). Los temas se asignarán por orden de llegada. Sin embargo, no
se sorprendan si les ofrezco “voluntariamente” presentar un artículo.

\vspace{-0.25cm}
\subsubsection*{Talleres}

Los laboratorios tienen como objetivos demostrar las estrategias de identificación utilizadas por los autores de los artículos
y su implementación en Stata o en R. Una semana antes de la entrega de los talleres, daré un taller demostrativo sobre un
artículo específico que sea relacionado al artículo que tendrán que replicar. Los talleres se realizarán en grupos de 3 o 4
personas y deberán entregar los siguientes archivos: (1) do file o R script, rmd; y (2) informe sobre el taller. Las orientaciones
sobre el laboratorio se harán una semana antes de su entrega.

\vspace{-0.25cm}
\subsubsection*{Ensayos}

Los ensayos individuales y grupales deberán de seguir las normas APA. Los ensayos grupales estarán conformados por 3
o 4 personas. La pregunta por contestar o enunciado a reflexionar lo proporcionaré con una semana de anticipación.. 

\subsection*{Honestidad e integridad académica}

A los estudiantes que se les encuentre copiando o plagiando se les asignará automáticamente una calificación de cero. Los
estudiantes tienen la responsabilidad de cumplir con los estándares de ética académica e integridad en todo su trabajo
académico. La violación de la ética académica se remitirá al Departamento de Economía Aplicada para su adjudicación y
puede estar sujeta a las sanciones correspondientes según lo definen las políticas de Universidad Centroamericana

\subsection*{Accesibilidad}

Si tiene una discapacidad documentada y anticipa acomodaciones en este curso, haga arreglos conmigo durante la primera
semana del cuatrimestre.

\section*{Bosquejo de las sesiones presenciales}
\label{sec:outline}

\subsection*{Unidades}

\begin{enumerate}
\item Introducción al concepto e importancia de la economía social y humana
\item La economía social y humana y su relación con los cambios institucionales
\item Interacciones sociales
\item Evidencia empírica en educación, mercado laboral, salud, pobreza y bienestar social
\item La economía social y humana en el contexto nicaragüense
\end{enumerate}

\newpage 
\section*{Referencias}
\begin{hangparas}{.25in}{1}
Acemoglu, D., Johnson, S., \& Robinson, J. (2005). Los orígenes coloniales del desarrollo comparativo: Una investigación empírica. \textit{Revista de Economía Institucional, 7}(13), 17-67.

Alianza Cooperativa Internacional de las Américas. (2007). \textit{ Diagnóstico del sector social de la economía en Nicaragua}. San José: Autor.  

Bandiera,O.,  Barankay, I., \& Rasul, I. (2005). Social preferences and the response to incentives: Evidence from personnel data. \textit{Quarterly Journal of Economics, 120}(3), 917-962. 

Conley, T., \& Udry, C. (2010). Learning about a new technology: Pineapple in Ghana. \textit{American Economic Review, 100}(1): 35-69. 

Coraggio, J. (2011). \textit{Economía social y solidaria: El trabajo antes que el capital.} Quito: Facultad Latinoamericana de Ciencias Sociales.  

Currie, J., \& Moretti, E. (2003). Mother’s Education and the intergenerational transmission of human capital: Evidence from college openings. \textit{Quarterly Journal of Economics, 118}(4), 1495-1532. 

De Sousa, B. (2011). \textit{Producir para vivir: Los caminos de la producción no capitalista.} México: Fondo de Cultura Económica.  

Dell, M. (2011). Los efectos persistentes de la mita minera en el Perú. \textit{Apuntes, 38}(68), 211-265. 

Díaz-Muñoz, G. (2015). \textit{Economías solidarias en América Latina}. Guadalajara: Instituto Tecnológico y de Estudios Superiores de Occidente.  

Garces, E., Thomas, D., \& Currie, J. (2002). Longer-term effects of head start. \textit{American Economic Review, 92}(4), 999–1012. 

Kim, H., Choi, S., Kim, B., \& Pop-Eleches, C. (2018). The role of education interventions in improving economic rationality. \textit{Science,} (362), 83-86. 

Malamud, O., \& Pop-Eleches, C. (2011). Home computer use and the development of human capital. \textit{Quarterly Journal of Economics, 126}(2), 987–1027. 

North, D. (2006). \textit{Instituciones, cambio institucional y desempeño económico}. México: Fondo de Cultura Económica. 

Nunn N., \& Wantchekon, L. (2011). The slave trade and the origins of mistrust in Africa. \textit{American Economic Review, 101}(7), 3221-3252. 

Organizacíon Internacional del Trabajo. (2015). \textit{Políticas públicas para la economía social y solidaria: hacia un entorno favorable}. Turín: Autor.  

Ostrom, E. (2000). \textit{El gobierno de los bienes comunes. La evolución de las instituciones de acción colectiva}. México: Fondo de Cultura Económica.

Ostrom, E. (2015). \textit{Comprender la diversidad institucional}. México: Fondo de Cultura Económica. 

Pop-Eleches, C., \& Urquiola, M. (2013). Going to a better school: Effects and behavioral responses.\textit{ American Economic Review, 103}(4), 1289-1324. 

Sen, A. (2009). \textit{La idea de la justicia}. Buenos Aires: Taurus. 

Thaler, R., \& Sunstein, C. (2008).\textit{ Un pequeño empujón. El impulso que necesitas para tomar mejores decisiones sobre salud, dinero y 
felicidad}.  Buenos Aires: Taurus. 
\end{hangparas}
\end{document}
